   %%%% Packages

\usepackage{amssymb}
\usepackage{amsfonts}
\usepackage{amsmath}
\usepackage{amsthm}

\usepackage{savesym}
\usepackage{bm}
\usepackage{algorithm}
\usepackage{algorithmic}
\floatname{algorithm}{Procedure}
\renewcommand{\algorithmicrequire}{\textbf{Input:}}
\renewcommand{\algorithmicensure}{\textbf{Output:}}

\savesymbol{comment}
\usepackage{easyReview}

\usepackage{comment}  %% block comments

\usepackage{color}
\newcommand{\vnote}[1]{\textcolor{red}{Vinod's note: {#1}}}
\newcommand{\leonote}[1]{\textcolor{red}{Leo's note: {#1}}}
\newcommand{\writeme}{\textcolor{red}{Write Me}}

\newcommand{\round}[1]{\lceil #1 \rfloor}
\newcommand{\floor}[1]{\lfloor #1 \rfloor}
\newcommand{\bigfloor}[1]{\big\lfloor #1 \big\rfloor}
\newcommand{\Bigfloor}[1]{\Big\lfloor #1 \Big\rfloor}

\newcommand{\anglebracks}[1]{\langle #1 \rangle}

\makeatletter
\newcommand\newtag[2]{#1\def\@currentlabel{#1}\label{#2}}
\makeatother

%
% defining the \BibTeX command - from Oren Patashnik's original BibTeX documentation.
\def\BibTeX{{\rm B\kern-.05em{\sc i\kern-.025em b}\kern-.08emT\kern-.1667em\lower.7ex\hbox{E}\kern-.125emX}}
    
    

\newtheorem{theorem}{Theorem}[section]
\newtheorem{definition}{Definition}[section]
\newtheorem{corollary}{Corollary}[theorem]
\newtheorem{lemma}[theorem]{Lemma}
\newtheorem{remark}{Remark}[section]

\newcommand{\calA}{\mathcal{A}}
   
   \newcommand{\olea}{\alpha}
    \newcommand{\oleb}{\beta}
    \newcommand{\olex}{x}
    \newcommand{\oleg}{\gamma}
    \newcommand{\veca}{\boldsymbol{\olea}}
    \newcommand{\vecb}{\boldsymbol{\oleb}}
    \newcommand{\vecx}{\mathbf{\olex}}
    \newcommand{\vecg}{\boldsymbol{\oleg}}
    
    \newcommand{\encd}[1]{[\![ #1 ]\!]}
    
    \newcommand{\getsr}{\xleftarrow{\$}}
    
    \newcommand{\ring}{\mathcal{R}}
    \newcommand{\Z}{\mathbb{Z}}
    \newcommand{\olering}{\Z_p}
    
    \newcommand{\KeyGen}{\mathsf{KeyGen}}
    \newcommand{\Encrypt}{\mathsf{Encrypt}}
    \newcommand{\Decrypt}{\mathsf{Decrypt}}
    
    \newcommand{\sk}{\mathsf{sk}}
    \newcommand{\pk}{\mathsf{pk}}
    \newcommand{\evk}{\mathsf{evk}}
    \newcommand{\ct}{\mathsf{ct}}
    
    \newcommand{\Eval}{\mathsf{Eval}}
    \newcommand{\EvalAdd}{\mathsf{EvalAdd}}
    \newcommand{\EvalAddPlain}{\mathsf{EvalAddPlain}}
    \newcommand{\EvalMultPlain}{\mathsf{EvalMultPlain}}
    
    \newcommand{\Sim}{\mathsf{Sim}}
    \newcommand{\Dis}{\mathcal{D}}
    \newcommand{\View}{\mathsf{View}}
    
    \newcommand{\RLWE}{\mathsf{RLWE}}
    
    \newcommand{\Hybrid}{\mathsf{Hybrid}}
    
    \newcommand{\VOLE}{\mathsf{VOLE}}
    \newcommand{\BOLE}{\mathsf{BOLE}}
    
    \newcounter{protocol}
    \makeatletter
    \newenvironment{protocol}[1][htb]{%
        \let\c@algorithm\c@protocol
        \renewcommand{\ALG@name}{Protocol}% Update algorithm name
        \begin{algorithm}[#1]%
    }{
        \end{algorithm}
    }
    \makeatother
    
    \newcommand{\sbline}{\\[.5\normalbaselineskip]}% small blank line