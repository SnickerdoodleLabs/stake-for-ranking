\section{Background}
Before we begin discussing the Snickerdoodle protocol it's worth diving into background into the current data economy and how the protocol aims to improve it. This section will address our problem statement, some terminology we will be using throughout the paper, and other solutions that have been proposed or built to try and address the problems we see.

\subsection{Problem Statement}
% User data is extracted but not owned by the user and data is handled unsafely
%   - Consent
%   - Compensation
%   - Transparency


This leads us to the Snickerdoodle finds the following problem:

$$\textit{Individuals are constantly having their data extracted from them but they don't own that data}$$

Large companies are constantly monitoring their users for data and these users don't have a good way to control their data. This observation and collection of people's data played a large factor in the shaping of the modern economy (CITE), even called surveillance capitalism by some (CITE). This has lead to a variety of negative consequences (TODO SOURCES PLUS FLESH OUT -- stuff about who gets the value and security, privacy, surveillance consequences, transparency, compensation, consent). 

The Snickerdoodle Protocol aims to help individuals control their own data.


\subsection{Existing Terminology}
As we get into this paper we'll be using a variety of terms that we will define here.
\subsubsection{Decentralization}
% - Permissionless 
% - Trustless
% - Available
\begin{definition}
\label{definition:Decentralization}
Decentralization: When control over a system is held by a group rather than a single authority.
\end{definition}

In order to prevent a single party from getting too much control in the data economy, including Snickerdoodle Labs, the Snickerdoodle Protocol will be built on top of a decentralized blockchain. 

An important distinction is that decentralization by itself isn't the ultimate goal. Rather we want the Snickerdoodle Protocol to be permissionaless, trustless, and available and the only known way to achieve these properties is by making the protocol decentralized.

\paragraph{Permissionless}
Anyone who follows the protocol's rules should be able to interact with the protocol. This means that Snickerdoodle Labs should not be able to decide which individuals are able to collect and share their data, choose what businesses are able to request data, and what developers are able to build on top of the protocol. 

The rules of the protocol will be determined by a Decentralized Autonomous Organization (DAO) which will create a decentralized way to manage the protocol (CITE). See sections \ref{section:DAO} and \ref{section:tokenDoa} for more details about the Snickerdoodle DAO.

\paragraph{Trustless}
No one in the system needs to rely solely on trust in order for the protocol to function.  Actors in the system shouldn't need to rely on Snickerdoodle Labs or any other actors in order to own their own data or buy insights.

It is worth pointing out that there is no such things as completely trustless and there are varying levels of trustlessness. Ideally, one can trust many mathematicians and engineers that the math and systems built will force the system to behave in the correct way. In the worst case a strong financial incentive can be relied on for the system to behave correctly. The Snickerdoodle Protocol will rely on both types of trustlessness and aim to update the protocol to use stronger forms of trustlessness over time.

\paragraph{Availability}
Actors in the system should be able to take feasible actions in the system in a reasonable amount of time. No one would use a system that can't be used when they want to use it. There's no point in individual's being able to own and control their own data if they aren't able to actually take any actions. 

\subsubsection{Data Safety}
% security
% Privacy

When describing data, we often say that it is $\textit{safe}$ in order to encompass all aspects of data safety. Safe data is data that is securely written, stored, and transmitted and accessed in a privacy preserving manner. This effectively means that only authorized people are able to access and know about the data.

\begin{definition}
\label{definition:DataSafety}
Data Safety: Data is considered safe if it is securely stored and privately viewed.
\end{definition}

\subsubsection{Data Subscribing}
% Data consuming entity
In the modern data economy there are many organizations that need data to gain insights about the world. In a world where data is owned by individuals, organizations wouldn't own and store individual data forever, rather they would pay to be granted temporary access to that data.

\begin{definition}
\label{definition:DataSubscriber}
Data Subscriber: A data subscribers is a data consuming entity that pays for temporary access to data
\end{definition}


\subsubsection{Data Terms}
% Wareshouing / Data Lake
% Collection / extraction
% Verifiability / autneticity / signatories
% Data freshnesses
Below we discuss other common terminology relating to data itself.

\paragraph{Data Warehousing}
%https://www.investopedia.com/terms/d/data-warehousing.asp#:~:text=Data%20warehousing%20is%20the%20secure,insight%20into%20the%20organization's%20operations.
Much like how a retailer keeps a large supply of stock on hand in a warehouse, modern businesses accumulate huge amounts of digital data in centralized data warehouses. These consolidated data warehouses serve as the data supply of the company from which it can extract valuable insight.

\paragraph{Data Mining}
Data mining is the process by which large data sets are collected and value is extracted from them. Many businesses employ data mining principles to index and extract value from their data warehouses.

\paragraph{Verifiability \& Authenticity}
Data is only as valuable as its trustworthiness. As demonstrated throughout the web3 space, one of the core applications of cryptography is authenticity. Through signing with trusted keys, data can be proved to be verified by a trusted authority. (expand more on signing)

\paragraph{Data Freshness}
Data freshness refers to how recent was the data collected. The more recent data has been collect to more useful it is (CITE?). For example, GIVE EXAMPLE (find info on Google search and CITE).

\subsubsection{Web3 Terms}
Below we discuss terminology that is often used in the web3 space.

\paragraph{Interoperability}
Data is interoperable if many different software applications can easily make use of that data. For example, if my music playlists were interoperable I could easily move them back and forth between Apple Music and Spotify. Interoperability is foundational to the Web3 space as users can move their assets easily among multiple different blockchains and developers can build their applications to use and modify existing tokens.

\paragraph{Key Management}
Key management is the process and policies by which cryptographic signing keys are stored and used. A cryptocurrency wallet software is an example of this, as it facilitates the storage and usage of the account keys on behalf of the user.

\paragraph{Signing}
Signing a foundational cryptographic function that allows a someone to prove that they have created or seen a message. This is really helpful for showing that a user has consented to an action (e.g., Alice signs a message sending Bob 1 Bitcoin and everyone else knows that Alice consented to that transfer).

% Signing
% Key Management

\subsection{Other Solutions}
There are a variety of different approaches and technology that aim to allow users to own their own data. In this section we discuss some of these approaches
\subsubsection{Policy Solutions}
%Follow GDPR + CPAA
%Warehousing / Data lake
Policy decisions such as GDPR in the European Union or the CCPA in California, attempt to regulate consolidated data warehouses and other types of centralized storage. These give individuals rights which allow them to control how their data is used (CITE). These laws a good step in giving individuals ownership over their data; however, these laws can be hard for people to interpret, for developers to build for, and for people to act on if the laws are broken (CITE + maybe give examples). The solution Snickerdoodle Labs aims to build aims to fix these problems by making a system that is compliant to these regulations, easy for developers to build with to allow their systems to be compliant, and easy for individuals to express their rights.


\subsubsection{Data Sharing Techniques}
%Perturbation Techniques
%   -DP
%Federated Learning
%Data Outsourcing
%   -We are making
There also exist a number of solutions that attempt to tackle the issues surrounding data-sharing. For example, perturbation techniques like differential privacy have shown promise in sharing noisy and/or anonymized data with limited value loss (CITE). Techniques such as federated learning and  multi-party-compute have been used to train models on distributed data sets(CITE). In addition, data outsourcing techniques have been employed to separate the management of data from its storage(CITE). All of these solutions are still yet to find practical applications for the most part and are hard for others to build on (CITE). Snickerdoodle Labs aims to combine the best of recent advancement to make it easy to develop these kinds of solutions. 

\subsubsection{Web3 Solutions}
% Data pools / unions
% Data set markets (Ocean)
The newness and distributed nature of web3 provide a natural way to explore data ownership and decentralizing the control of data. Ceramic creates a way to create link existing databases in a decentralized manner and manage their identity (CITE). Ocean Protocol creates a data set market by allowing people to sell access to data sets and bring compute to data (CITE). While these projects are inspired and create new ways to interact with data in a decentralized way, they don't address the problem of allowing individuals to own and control their data. 