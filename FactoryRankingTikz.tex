\begin{figure}[!ht] 
    \centering
    % Define block styles
    \tikzstyle{factoryblock} = [rectangle, draw, fill=blue!20, 
    text width=8em, text centered, rounded corners, minimum height=4em]
    \tikzstyle{instance} = [circle, draw, fill=violet!20, 
        text width=6em, text centered, rounded corners, minimum height=4em]
    \tikzstyle{line} = [draw, -latex']
    
    \begin{tikzpicture}[node distance = 2cm, auto]
        % Place nodes
        \node [factoryblock,  node distance=5cm] (FACTORY) {Contract Factory \& Global Ranking};
        \node [instance, below of=FACTORY,  node distance=3cm] (CONTRACT2) {Content Object 2};
        \node [instance, left of=CONTRACT2,  node distance=4cm] (CONTRACT1) {Content Object 1};
        \node [instance, right of=CONTRACT2,  node distance=4cm] (CONTRACT3) {Content Object N};
        % Draw edges
        \path [line] (FACTORY) -| node [near start, above] {deploy} (CONTRACT1);
        \path [line] (CONTRACT1) -- node [near end, left] {claim rank} (FACTORY);
        \path [line] (FACTORY) -- node [right] {deploy} (CONTRACT2);
        \path [line] (FACTORY) -| node [near start, above] {deploy} (CONTRACT3);
        \path [line] (CONTRACT3) -- node [near end, right] {claim rank} (FACTORY);
    \end{tikzpicture}
    \caption{Stake for ranking works well with factory patterns since the contract factory serves as a natural location to store global ranking data structures. It also allows for programmatic exclusion of content not deployed through the factory itself which can be useful for certain applications.}
    \label{fig:FactoryRanking}
  \end{figure}