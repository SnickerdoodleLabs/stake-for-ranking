\begin{figure*}[!htbp] 
    \centering
    \begin{tikzpicture}[
      scale=0.75,
      start chain=1 going below, 
      start chain=2 going right,
      node distance=1mm,
      desc/.style={
        scale=0.75,
        on chain=2,
        rectangle,
        rounded corners,
        draw=black, 
        very thick,
        text centered,
        text width=8cm,
        minimum height=12mm,
        fill=purple!60
        },
      it/.style={
        fill=violet!20
      },
      level/.style={
        scale=0.75,
        on chain=1,
        minimum height=12mm,
        text width=2cm,
        text centered
      },
      every node/.style={font=\sffamily}
    ]
    
    % Levels
    \node [level] (Level 5) {Presentation Layer};
    \node [level] (Level 4) {Identity Layer};
    \node [level] (Level 3) {Events Layer};
    \node [level] (Level 2) {Execution Layer};
    \node [level] (Level 1) {Permissions Layer};
    \node [level] (Level 0) {Data Layer};
    
    % Descriptions
    \chainin (Level 5); % Start right of Level 5
    % application layers
    \node [desc, it] (Form) {Application and Form Factor Logic};
    \node [desc, it, continue chain=going below] (Identity) {Signature Verification and Identity Creation};
    % protocol layers
    \node [desc] (Events) {Environment and Protocol Event Detection};
    \node [desc] (Execution) {Query Interpretation and Execution};
    \node [desc] (Permissions) {Granular Attributed-Based Access Control};
    \node [desc] (Data) {Indexing/Storage/Synchronization of Web2/Web3 Data};
    
    \end{tikzpicture}
    \caption{Logical structure of a protocol data wallet.}
    \label{fig:DataWalletStructure}
  \end{figure*}