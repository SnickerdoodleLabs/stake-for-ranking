\section{Stake for Ranking} 
\label{section:ProtocolDescription}

\subsection{Preliminaries}
Stake-for-ranking attempts to address some of the shortcomings of closed-source, proprietary recommender systems by removing the need for a centralized party to curate content and its relative importance to an end user and allow for the natural and direct alignment of incentives between content authors and their audiences. It contributes:
\begin{itemize}
    \item Transparent ranking mechanics due to auditable on-chain logic
    \item Sybil resistance against spam listings due to economic costs of entering recommender system and curating attribute weights
    \item Can be coupled with ERC-7529 \cite{chapman2023erc7529} for enhanced content spoofing protection and DNS/domain-based moderation
    \item enhanced end-user privacy due to decentralized content delivery
\end{itemize}

\subsection{Components of the Primitive}
\subsubsection{Content Objects}
A content object is an on-chain asset (a smart contract) owned or operated by a content author and can be an ERC-721 NFT, an ERC-20 token, or any other contract-based asset type. It is also assumed that the content object is some kind of asset that an end-user will be transacting with directly, like paying into a contract to mint an nft or fungible token (thus the need to attract end users to your content object in the first place). 

A content object will possess an interface defined in section \ref{InterfaceDefinition} and a member variable mapping an attribute string (i.e. a human-readable tag) to a weight: 

\begin{equation}
    attribute_i \rightarrow w_i
\end{equation}

This mapping represents an infinite dimensional sparse vector of weights. The weight vector is limited to a fixed, finite number of non-zero entries, and the magnitude of the non-zero entries is proportional to the amount of token the content author wishes to stake to that row’s associated keyword as discussed later in section \ref{PrimitiveMechanics}. For programming convenience in the context of the Ethereum Virtual Machine \cite{wood2014ethereum} (which works nicely with unsigned integers) weights, $w_i$ are taken to be non-negative integer values. 

\subsubsection{Global Ranking Linked List}

\begin{figure}[!ht] 
    \centering
    % Define block styles
    \tikzstyle{daoblock} = [diamond, draw, fill=red!20, 
    text width=6em, text centered, rounded corners, minimum height=4em] 
    \tikzstyle{factoryblock} = [rectangle, draw, fill=blue!20, 
    text width=6em, text centered, rounded corners, minimum height=4em]
    \tikzstyle{instance} = [circle, draw, fill=violet!20, 
        text width=6em, text centered, rounded corners, minimum height=4em]
    \tikzstyle{actorblock} = [rectangle, draw, fill=green!20, 
        text width=6em, text centered, rounded corners, minimum height=4em]
    \tikzstyle{line} = [draw, -latex']
    
    \begin{tikzpicture}[node distance = 2cm, auto]
        % Place nodes
        \node [factoryblock,  node distance=5cm] (FACTORY) {Contract Factory \& Global Ranking};
        \node [instance, below of=FACTORY,  node distance=3cm] (CONTRACT2) {Content Object 2};
        \node [instance, left of=CONTRACT2,  node distance=4cm] (CONTRACT1) {Content Object 1};
        \node [instance, right of=CONTRACT2,  node distance=4cm] (CONTRACT3) {Content Object ...};
        % Draw edges
        \path [line] (FACTORY) -| node [near start, above] {deploy} (CONTRACT1);
        \path [line] (CONTRACT1) -- node [near end, left] {claim rank} (FACTORY);
        \path [line] (FACTORY) -- node [right] {deploy} (CONTRACT2);
        \path [line] (FACTORY) -| node [near start, above] {deploy} (CONTRACT3);
        \path [line] (CONTRACT3) -- node [near end, right] {claim rank} (FACTORY);
    \end{tikzpicture}
    \caption{Stake for ranking easily works with factory patterns since the contract factory serves a a natural location to store global ranking data structures. It also allows for programmatic exclusion of content not deployed through the factory itself which can be useful for certain applications.}
    \label{fig:FactoryRanking}
  \end{figure}
The smart contract architecture is assumed to be based on a factory pattern, although it is not strictly necessary. A contract factory is a natural location to host the data structures responsible for storing the global ranking of content objects with respect to each tag. It also allows for the ranking structure to exclude the listing of content objects that were not deployed through standard template that would be defined by a contract factory. 

Global ranking can be stored in a mapping variable that maps from an attribute string to a doubly linked list:

\begin{equation}
    attribute_i \rightarrow \{(2^{256}-1) \leftrightarrow w_{i,...} \leftrightarrow w_{i,n+1} \leftrightarrow w_{i,n} \leftrightarrow w_{i,n-1} \leftrightarrow w_{i,...} \leftrightarrow 0\}
\end{equation}

The weight, $w_{i,n}$, belongs to the $n^{th}$ highest ranked content object for the $i^{th}$ attribute such that $w_{i,n} < w_{i,n+1}$. Notice that the head of any attribute's linked list is the maximum value of a $uint256$ and the tail value is $0$, thus $0 < w_{i,n} < (2^{256}-1); \forall i,n$. 

Linked lists are ideal for tracking dynamically allocated content rankings in a smart contract setting due to the constant complexity of insertion and deletion of a list entry. Furthermore, entries can easily be traversed in a paginated fashion in either ascending or descending order. 

\subsubsection{Likeness metric}

A content-based recommender system must prescribe a metric by which a content object can be determined to be relevant to an end user. This is an off-chain function that takes as input a user's historical state vector and content object's weight vector and returns a score. We assume in this work that a perfect match would be normalized to $1$ and a perfect miss would be $0$.

\subsection{Interface Specifications}
\label{InterfaceDefinition}

\subsubsection{Content Objects}

Content objects must have functions for staking and destaking economic value with respect to an attribute that is relative to the content. 

\subsubsection{Global Ranking Linked List}

Client-side software will benefit from the presence of a single contract where global ranking information can be obtained. The global ranking contract will mostly be used for reading ranking data. 

register a new attribute

\subsection{Primitive Mechanics}
\label{PrimitiveMechanics}
A representation vector of a content object is constructed by assigning weights to relevant tags. 
The weight attributable to a tag is proportional to the amount of token staked w.r.t that tag
Candidate weighting cost function: 

\begin{equation}
    stake_i = 1.0001^{w_i}
\end{equation}

As the desired target weight increases linearly, the amount of stake required to achieve that target weight increases exponentially, disincentivizing frivolous weighting of genuinely unrelated tags and also normalizing the weighting power of large token holders to make it more difficult to grossly overweight high-demand tags. 
A non-zero minimum weight could be imposed to further disincentivize applying small staking amounts to tags that are not actually relevant to a content object

There is economic cost to curating the representation vector of a content object
Assigning weight to tags that are not actually relevant to your content object locks up capital 
When a content author stakes token to a given tag/keyword in their content object, this results in a registration of that content object in a global linked list that tracks all content objects staking that particular tag; the content object with the most stake for a given tag takes the head slot, second most gets the second slot, and so on. 
Thus, the global ordering for a tag effectively forms an open order book for ranking of content associated with a given tag. Content authors and end users both have direct insight into the relative ranking of content for a given tag/attribute. 

Staking economic value does not result in rewards issued directly from the protocol itself to the depositor, instead it boosts the discoverability and visibility of a content object in a marketplace setting, leading to greater likelihood of end user transactions and thus revenue. Depositors can rearrange their stake among different tags as they search the state space for the correct tags for their target audience and can reclaim all of their stake and exit the market if necessary. This is an important distinction from existing staking models. 
Popular tags (i.e. tags that are relevant to the largest number of users) will cost more for a higher number of impressions while tags associated with more specialized audiences will likely be cheaper.

The economic cost of attribute staking coupled with ERC-7529 ensures that spoofing content to attack the recommender system is difficult and costly and makes it simple to implement client-side verification. Additionally, staked tokens are subject to forfeiture if the token holder community deems the content object to be detrimental to the associated marketplace. Therefore, stake for ranking has a natural resistance to sybil attacks from bad authors and incentives to construct an accurate representation vector for their content since it requires capital to be locked in proportion to the representation weights. 