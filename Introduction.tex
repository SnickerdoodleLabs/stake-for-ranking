\section{Introduction}

The design, implementation, and stability analysis of cryptoeconomic primitives \cite{horne2018crypto} has been an intensive area of research since the emergence of smart contract networks like Ethereum \cite{buterin2014next}. Some prominent examples include Proof of Stake \cite{quantum2011bitcoin} (now the consensus mechanism securing Ethereum itself), tokenization (particularly stable coins backed by digital or traditional assets), curve bonding \cite{graphBondingCurve}, token curated registries \cite{Goldin2018TCR} (TCRs), yield staking, and distributed autonomous organizations \cite{merkle2016daos} (DAOs). 

In short, a cryptoeconomic primitive is a type of economic game in which the presence of a programmable token asset (which may or may not have a fixed supply) is a prerequisite for the proper functioning of the game. Proper functioning encompasses the initial and ongoing incentive alignment of the participants so that a stable (quasi)equilibrium of the game or market can be achieved without a centralized operator. The list of cryptoeconomic primitive examples given have sought to solve problems in finance, governance, and information asymmetry, as well as other areas. 

This work proposes a new primitive, stake for ranking, which aims to serve as the core of a self-governing, content-based recommender system. This can be seen as an addition to the tool kit of curation market primitives which, among other things, seek to provide verifiable information to all parties on equal footing. As will be outlined, the token mechanics of stake for ranking also share similarities both with TCRs and yield staking. 