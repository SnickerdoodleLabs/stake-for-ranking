Value generation in the digital age has been largely driven by the extraction and ownership of virtual assets. 
        -data mining
        -data collection
        -existing data economy
-highlight current value + state of digital assets

While recent innovation has led to a large push towards decentralization in the digital world and the creation of new virtual asset classes, the most valuable digital asset has always been data.
        -tie back to blockchain creates digital assets 
        - information + insights in general has been most valuable 
-Data / information / insights are the hightest value

Historically, data has only been commercially valuable in the aggregate, leaving those who create it unable to benefit from their production.
        - focus on data by itself isn't valuable
        - data individuals create they can get value out of
-Right now individuals aren't getting their fair share

In addition, increasing privacy concerns and the lack of regulation around data collection has led to increasing public distrust of technology as a whole. 
-Tie to real world impact + shady data brokers 

The Snickerdoodle Protocol aims to solve these problems by allowing individuals to collect, control, and own their own data by creating the decentralized infrastructure required to turn data into a self-sovereign asset.
-How we solve


-WHAT + (tiny)why is snickerdoodle labs doing?
Snickdoodle Labs is creating the Snickerdoodle protocol to enable permissionless federated data markets in which user's own and retain custody of their personal data and entities wishing to gain insights from user data must obtain consent and remunerate end user's for their participation.

The Snickerdoodle protocol enables a permissionless individualized / self-sovereign data economy in which individuals control the extraction, ownership, and usage of their data insights.

-What is the state of the world today + what the problem is?
     -Existing data economy, data broker, extraction+ownership (extraction of natural resources)
     -Individuals are creating lots of valuable data but not getting value out of it

-How is it done today, and what are the limits of current practice?
    - Value + power in the extraction BUT individuals creating data aren't getting value
    - Privacy
    - Large central data brokers
    - Progress in a good way (Apple+Google private federated learning on mobile), it's closed off and they benefit
    - Make data asset on BC but geared towards aggregate data but not individuals
    - BC identities

-What are you trying to do? Articulate your objectives using absolutely no jargon.
    - Open protocol to turn data into asset and generate value
    - Want protocol take form factor that non-technically sophisticated end user can participate
    - Want lower privacy burden on companies so don't have to deal with regulations (operational risk reduction)
    - Why wouldn't you use it?

-What is new in your approach and why do you think it will be successful?
    - permissionless - DAO control data market
    - focus on individuals
    - Federated queirying, learning, DP
    - Consent models
    - Define data as an asset
    - "Data Wallet"
    - Choose what data associated with which identities + can KYC with my own data
     
    
    - Why wouldn't you use it?
    - Incentive for individuals to provide + claim value from data. Upward mobility in data economy
    - End user will be able to easily use  
    - Precedent for this - web3 users growing at a rate like internet in 90s + these users are aware of current bad data economy structure

-What are the risks?
    - Adoption (hard to get people to use a new product -- 2 sided market risk)
    - Ambiguous definition of data as an asset -- need to find way to add use and lead to adoption
    - Shifting an established, hundred-billion data economy
    - Relying on security of cryptographic models that might not hold up
    - Privacy concerns of blockchain
    - Will we make good incentive scheme (compensates all parties fairly to encourage good behavior + not predatory)
    - Equity (assets in general aren't distributed equitably and we are making a new one)


-Who cares? If you are successful, what difference will it make?
    - Individuals - privacy + value from data
    - Infrastructure providers bc new market
    - Businesses reach more users with less risk
    - Fix a broken data economy


