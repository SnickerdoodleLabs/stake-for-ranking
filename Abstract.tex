\section{Abstract}

The Synamint Protocol aims to enable an open, equitable, and consent-driven federated data market in which individuals will 
collect, own, and authorize the use of their own data. Data is most valuable in the aggregate so that 
it can be leveraged to build predictive models; this results in a data market where centralized aggregators hold 
the only economically valuable role. In this structure, individual users who actually generate data are often 
left with little to no benefit from the value of their data. 

As more and more privacy violations and data breaches continue to occur on a larger scale, it has become apparent that the
centralization of data poses a serious threat to both the individual and the community as a whole as it results in a
single point of failure for malicious actors to focus their efforts. Technology companies, like Apple and Google, have
produced software designed to provide more privacy forward data collection through federated learning and multi-party 
computation models optimized to run on hardware devices that these companies produce. However, these solutions are 
neither auditable nor open, and therefore still leave the end user removed from the value they generate. In addition, 
existing decentralized solutions often focus primarily on the sale of aggregated data sets. The protocol described in this 
whitepaper aims to enable the extraction and collection of new data in real time via edge computing techniques leveraging
a permissionless blockchain as an orchestration layer and control plane. 

This protocol will alleviate these concerns by turning user data into an individualized asset that can be leveraged 
without the need for an initial consolidation step. User-controlled localized ”Data Wallets” will provide an easily usable form 
factor to the end user, while a business-oriented ”Insights Service” will provide analytical and visualization services to entities who wish to extract 
intelligence from the data itself. By providing consent-driven ownership and granular access control to the end-user, we allow them to gain greater sovereignty over their 
digital identity, while freeing businesses from the overhead of regulatory compliance with regard to user data. Our solution will be 
permissionless, community-governed, and self-sovereign, and will provide an alternative to the data economy of today. Internet users 
are already recognizing the need for decentralization in the information economy, as demonstrated by the fact that the Web3 user base 
is growing at a rate comparable to the internet in the 1990s. By properly incentivizing these users, allowing businesses to reach more 
users with less risk, and creating an open protocol for decentralized infrastructure providers, the protocol will facilitate 
a data economy that benefits all. 