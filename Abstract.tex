\section{Abstract}

The Snickerdoodle Protocol aims to enable an open and equitable federated data market in which individuals will collect, own, and authorize the use of their own data. Typically today data is only valuable in the aggregate, resulting in a data market where aggregation is the only economically valuable role. The individual users who actually own and extract this data are often left with little to no value as a result. This has led to the widespread consolidation of data between a minority of large entities who have come to possess a significant economic advantage as a result. As more and more privacy violations and data breaches continue to occur on an increasingly larger scale, it has become apparent that the centralization of data poses a serious threat to both the individual and the community as a whole. While efforts are being made to provide more privacy-forward data collection through federated learning models, existing solutions are not open and equitable, lacking transparency and again leaving the end user removed from the value they generate. In addition, existing decentralized solutions often focus primarily on the sale of aggregated data sets. Snickerdoodle aims to enable the extraction and collection of new data in real time at the edge.

The Snickerdoodle Protocol will alleviate these concerns by turning user data into an individualized asset that can be leveraged without the need for consolidation. User-controlled localized "Data Wallets" will provide an easily usable form factor to the end user, while a business-oriented "Insights Service" will provide analytic services to entities who wish to extract value from the data itself. By shifting control of security and privacy to the user, we allow them to gain greater sovereignty over their digital identity, while freeing businesses from the overhead of regulatory compliance with relation to user data. Our solution will be permissionless, community-governed, and self-sovereign, and will provide an alternative to the feudal data economy of today. Internet users are already recognizing the need for decentralization in the information economy, as demonstrated by the fact that the web3 userbase is growing at a rate comparable to the internet in the 1990s. By properly incentivizing these users, allowing businesses to reach more users with less risk, and creating an open protocol for decentralized infrastructure providers, the Snickerdoodle Protocol will facilitate a data economy that benefits all. 